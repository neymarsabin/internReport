\section{Requirement Collection}
\noindent
When it comes to any type of project, requirement collection plays a key role. Requirements
collection is not only important for the project, but it is also important for the project
management function. Although requirements collection looks quite straightforward,
surprisingly, this is one of the project phases where most of the projects start with the wrong
foot.After a bit of research and discussion that we made with our seniors, we went with Rapid Application Development(RAD) approach among all other requirement collection procedures. We had couple of RAD session where we discussed about the leaves management system, its functionalities and what should its requirements be? The participants of the RAD session were 2 of the full stack developers (including me), a senior software engineer, and a CEO of the organization, who happens to be the client of our application.

\section{Requirement Analysis}
The software requirements are description of features and functionalities of the target system.
Requirements convey the expectations of users from the software product. The requirements
can be functional and non-functional. Leaves Management System also has two types of
requirements; functional and non-functional. Both of these requirements are listed and explained
blow.

\subsection{Functional Requirements}
In order to build a system that is durable and well-functioning, functional analysis is one of
the most important among all other system requirements. It has a major impact on system
feasibility. Leaves Management System also has some functional requirements which are
listed as follows.

  \begin{enumerate}
  \item{New user should be able to register/signup.}
  \item{Users should be able to login to the system.}
  \item{Admin should be able to maintain user roles I.e. Employee/Manager.}
  \item{Admin should be able to create teams with managers along with their employee.}
  \item{Users(Employee/Manager) should be able to request leave.}
  \item{Managers should be able to approve/decline leaves requested by employee in their team.}
  \end{enumerate}

 \subsection{Non-Functional Requirements}
  Non-functional requirements are secondary requirements of a system. Usually for web base
application, its performance, scalability, availability, maintainability, recovery etc. are some
of non-functional requirements that has to be address properly for its durability and optimized
performance. Similarly Leaves Management System also has some non-functional
requirements which are listed as follows.
\newline

\begin{enumerate}
\item{System should have a good response time for any request that are made.}
\item{System should also be scalable.}
\item{System should have high availability.}
\item{System should be maintainable as its complexity increases.}
\end{enumerate}
  


\section{Feasibility Study}
A feasibility study is used to determine the viability of an idea, such as ensuring a project is
legally and technically feasible as well as economically justifiable. It tells us whether a
project is worth the investment, in some cases, a project may not be doable. There can be
many reasons for this, including requiring too many resources, which not only prevents those
resources from performing other tasks but also may cost more than an organization would
earn back by taking on a project that isn’t profitable. Leaves Management System has also
passed through some of the feasibility analysis which are listed as follows.

\subsection{Technical Feasibility}
Leaves Management System is a tire application and is built in web bases requiring web
technologies to be used effectively. The entire application is built on server-side web
framework Ruby on Rails and client-side programming language framework ReactJS which
are both open source software packages. It uses postgres database which also happens to be
open source database management system. The availability of above mentioned technologies
are high. They are used by many big software companies proving its maturity. The software
team at Codyssey Web Nepal is expertized and is capable of handling the entire system
regarding its maintenance, scalability etc.

\subsection{Operational Feasibility}
It is the measure of how well a proposed system solves the problems and takes advantages of
the opportunities identified during the scope definition and problem analysis phases. And
how well it satisfies the system requirements identified in the requirement analysis phase.
Potential users of the system are familiar with the website navigation and handling. Hence
training up to necessary level would be easy. The implementation of the system in the
internet can be easily managed, and the security issues needs to be addressed in network level
or else in the application levels. Ruby on Rails and ReactJs both supports object-oriented
development approaches so that well defined design can maintain the smooth run and the
flexibility of the proposed system. User access levels will be set and the system will only
allow privileged users. Authentication, Authorization and Audit procedures will be facilitated
to the system administrators.

\subsection{Economic Feasibiliy}
Economic feasibility is the most frequently used method for evaluating the effectiveness of a new
system. More commonly known as cost/benefit analysis, the procedure is to determine the benefits
and savings that are expected from a candidate system and compare them with costs. If benefits
outweigh costs, then the decision is made to design and implement the system.

\begin{enumerate}
\item{Benefits}

  \begin{enumerate}
    \item{The system has better efficiency and effectiveness in the maintenance of personal information details and calculations in the leave system thus reducing human errors.}
    \item{It has better employee motivation and flexibility due to improved efficiency in
      dealing with user friendly interfaces.}
    \item{It reduces labor cost as fewer employees would be needed in entering the data
      into the system and processing the information.}
    \item{It has better efficiency in applying leaves, obtain their leave information quickly and obtain other basic details.}
    \item{Data redundancy and security issues is lessened due to improved backups and security features in the system.}
  \end{enumerate}

\item{Cost-Benefit Analysis}
  We can divide the cost in to the few categories.
  \begin{enumerate}
  \item{Development and Purchasing Costs.}
    \begin{enumerate}
    \item{Hardware (Currently the got enough hardware facilities)}
    \item{Software}

      \begin{enumerate}
      \item{Operating System (Already installed and no need to change)}
      \item{Main software need to build the system (Open source software - Free)}
      \item{Other software (Total cost*25\%)}
      \end{enumerate}
        
    \end{enumerate}
  \item{Installation and Data entry Cost}

    \begin{enumerate}
    \item{Install the system}
    \item{Training the staff}
    \item{Enter previous data into system}
    \end{enumerate}
    
  \item{Operational Costs}
    \begin{enumerate}
    \item{Maintenance of the system}
    \item{Upgrading the software}
    \end{enumerate}

  \item{Personnel Costs}
    \begin{enumerate}
    \item{Maintenance staff cost}
    \item{Upgrading the software}
    \end{enumerate}
    
  \end{enumerate}
\end{enumerate}

\bfseries{Projected development cost for proposed system}
Expenses